%%%
% Config.tex
% Ficher de configuration du PV
%%%

% L'acronyme de l'instance en cours, sans la numérotation ex: CA, CM, ...
\newcommand{\instance}{CM}

% Compteur de l'année d'opération de la FEUS, seulement changer le dernier chiffre.
\newcounter{annee}
\setcounter{annee}{26}
% Compte de l'année en texte. Seulement changer le dernier champs.
\newcommand{\anneeaulong}{vingt-sixième}
% Utiliser/Afficher l'année dans le PV
\newbool{showYear}
%\booltrue{showYear}
\boolfalse{showYear}

% Compteur de la session de l'année d'opération de la FEUS, seulement changer le dernier chiffre.
\newcounter{session}
\setcounter{session}{3}
% Compte de la session en texte. Seulement changer le dernier champs.
\newcommand{\sessionaulong}{troisième}
% Utiliser/Afficher l'année dans le PV
\newbool{showSession}
%\booltrue{showSession}
\boolfalse{showSession}

% Compteur de l'instance de la session en cours, seulement changer le dernier chiffre.
\newcounter{instance}
\setcounter{instance}{2}
% Compte de l'instance en texte. Seulement changer le dernier champs.
\newcommand{\instanceaulong}{deuxième}

% Style: contrôler si les styles sont soulignés ou pas. Décommenter booltrue pour activer ou boolfalse pour désactiver.
\newbool{useUnderline}
\booltrue{useUnderline}
%\boolfalse{useUnderline}

% Date de la réunion
\date{16 mars 2016}

% Heure de la convocation. Simplement changer le dernier champs.
\newcommand{\heure}{17h30}

% Journée de l'instance. Simplement changer le dernier champs.
\newcommand{\jour}{mercredi}

% Local de l'instance. Simplement changer le dernier champs.
\newcommand{\local}{D2-1060}

% Session de l'instance. Simplement changer le dernier champs.
\newcommand{\session}{H16}
